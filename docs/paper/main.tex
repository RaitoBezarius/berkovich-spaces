% Options for packages loaded elsewhere
\PassOptionsToPackage{unicode}{hyperref}
\PassOptionsToPackage{hyphens}{url}
%
\documentclass[
]{article}
\usepackage{lmodern}
\usepackage{amssymb,amsmath}
\usepackage{ifxetex,ifluatex}
\ifnum 0\ifxetex 1\fi\ifluatex 1\fi=0 % if pdftex
  \usepackage[T1]{fontenc}
  \usepackage[utf8]{inputenc}
  \usepackage{textcomp} % provide euro and other symbols
\else % if luatex or xetex
  \usepackage{unicode-math}
  \defaultfontfeatures{Scale=MatchLowercase}
  \defaultfontfeatures[\rmfamily]{Ligatures=TeX,Scale=1}
\setmathfont{latinmodern-math.otf}
\fi
% Use upquote if available, for straight quotes in verbatim environments
\IfFileExists{upquote.sty}{\usepackage{upquote}}{}
\IfFileExists{microtype.sty}{% use microtype if available
  \usepackage[]{microtype}
  \UseMicrotypeSet[protrusion]{basicmath} % disable protrusion for tt fonts
}{}
\makeatletter
\@ifundefined{KOMAClassName}{% if non-KOMA class
  \IfFileExists{parskip.sty}{%
    \usepackage{parskip}
  }{% else
    \setlength{\parindent}{0pt}
    \setlength{\parskip}{6pt plus 2pt minus 1pt}}
}{% if KOMA class
  \KOMAoptions{parskip=half}}
\makeatother
\usepackage{xcolor}
\IfFileExists{xurl.sty}{\usepackage{xurl}}{} % add URL line breaks if available
\IfFileExists{bookmark.sty}{\usepackage{bookmark}}{\usepackage{hyperref}}
\hypersetup{
  pdftitle={Formalization of Ostrowski theorems in Lean theorem prover},
  pdfauthor={Ryan Lahfa,1,; Julien Marquet,1; Hadrien Barral1},
  hidelinks,
  pdfcreator={LaTeX via pandoc}}
\urlstyle{same} % disable monospaced font for URLs
\setlength{\emergencystretch}{3em} % prevent overfull lines
\providecommand{\tightlist}{%
  \setlength{\itemsep}{0pt}\setlength{\parskip}{0pt}}
\setcounter{secnumdepth}{5}

\usepackage[ruled, french, frenchkw]{algorithm2e} 
\usepackage{turnstile}
\usepackage{ebproof}
\usepackage{amssymb, upgreek}
\usepackage{color}
\definecolor{keywordcolor}{rgb}{0.7, 0.1, 0.1}   % red
\definecolor{commentcolor}{rgb}{0.4, 0.4, 0.4}   % grey
\definecolor{symbolcolor}{rgb}{0.0, 0.1, 0.6}    % blue
\definecolor{sortcolor}{rgb}{0.1, 0.5, 0.1}      % green
\definecolor{errorcolor}{rgb}{1, 0, 0}           % bright red
\definecolor{stringcolor}{rgb}{0.5, 0.3, 0.2}    % brown

\usepackage{listings}
\def\lstlanguagefiles{lstlean.tex}
\lstset{language=lean}
\usepackage{mathtools}
\usepackage{stmaryrd}

\DeclarePairedDelimiter\abs{\lvert}{\rvert}%
\DeclarePairedDelimiter\norm{\lVert}{\rVert}%
\DeclarePairedDelimiter\ceil{\lceil}{\rceil}%
\DeclarePairedDelimiter\floor{\lfloor}{\rfloor}%

\DeclareMathOperator*{\card}{card}%
\DeclareMathOperator*{\argmin}{argmin}%
\DeclareMathOperator*{\Mat}{Mat}%
\DeclareMathOperator{\Vol}{Vol}%
\DeclareMathOperator{\msucc}{succ}%
\DeclareMathOperator{\pgcd}{pgcd}%
\DeclareMathOperator{\ppcm}{ppcm}%
\DeclareMathOperator{\Ker}{Ker}%
\DeclareMathOperator*{\Vect}{Vect}%
\DeclareMathOperator{\rref}{rref}%
\DeclareMathOperator{\rg}{rg}%
\DeclareMathOperator{\lfp}{lfp}%
\DeclareMathOperator{\im}{Im}%

\usepackage{amsthm}
\newtheorem{theorem}{Theorem}
\newtheorem{lemma}{Lemma}
\newtheorem{corollary}{Corollary}[lemma]
\newtheorem{definition}{Definition}

\newcommand{\PR}{\mathbb{P}}
\newcommand{\E}{\mathbb{E}}
\newcommand{\N}{\mathbb{N}}
\newcommand{\R}{\mathbb{R}}
\newcommand{\Z}{\mathbb{Z}}
\newcommand{\Q}{\mathbb{Q}}
\newcommand{\K}{\mathbb{K}}
\newcommand{\M}{\mathcal{M}}
\newcommand{\F}{\mathbb{F}}
\newcommand{\class}[1]{\mathcal{C}^{#1}}
\newcommand{\ie}{\text{i.e. }}

\newcommand{\application}[5]{%
        \begin{array}{ccccl}%
        #1 & : & #2 & \to & #3 \\ %
         & & #4 & \mapsto & #5 \\ %
        \end{array}%
}%

\newtoks\rowvectoks
\newcommand{\rowvec}[2]{%
  \rowvectoks={#2}\count255=#1\relax
  \advance\count255 by -1
  \rowvecnexta}
\newcommand{\rowvecnexta}{%
  \ifnum\count255>0
    \expandafter\rowvecnextb
  \else
    \begin{pmatrix}\the\rowvectoks\end{pmatrix}
  \fi}
\newcommand\rowvecnextb[1]{%
    \rowvectoks=\expandafter{\the\rowvectoks&#1}%
    \advance\count255 by -1
    \rowvecnexta}

\newcount\colveccount
\newcommand*\colvec[1]{
        \global\colveccount#1
        \begin{pmatrix}
        \colvecnext
}
\def\colvecnext#1{
        #1
        \global\advance\colveccount-1
        \ifnum\colveccount>0
                \\
                \expandafter\colvecnext
        \else
                \end{pmatrix}
        \fi
}

% Swap the definition of \abs* and \norm*, so that \abs
% and \norm resizes the size of the brackets, and the 
% starred version does not.
\makeatletter
\let\oldabs\abs
\def\abs{\@ifstar{\oldabs}{\oldabs*}}
%
\let\oldnorm\norm
\def\norm{\@ifstar{\oldnorm}{\oldnorm*}}
\makeatother
\usepackage[style=alphabetic,]{biblatex}
\renewcommand*{\bibfont}{\small}
\addbibresource{./Formalization.bib}
\addbibresource{./Berkovich.bib}


\title{Formalization of Ostrowski theorems\\
in Lean theorem prover}
\author{Ryan Lahfa\textsuperscript{$\dagger{}$,1,*} \and Julien Marquet\textsuperscript{$\dagger{}$,1} \and Hadrien Barral\textsuperscript{1}}
\date{}

\usepackage{onecolceurws_pandoc}

\begin{document}
\maketitle
\begin{abstract}
Ostrowski theorems provide classification of all absolute values in
certain fields and lies at the foundations of Berkovich space theory. In
particular, over \(\Q\), all absolute values are either the trivial, the
usual or a \(p\)-adic. This statement entirely determines the Berkovich
spectrum of integers.

We formalize Ostrowski theorems in the Lean theorem prover, in two
attempts, one aiming to understand the challenges and determining a
reachable generalization target. The second attempt reaches this target
and shows everything the first attempt does in a simpler and cleaner
way. Following this road, we identify low-hanging fruits missing the
Lean mathematical library and develop a self-contained reusable general
theory to formalize Ostrowski theorems in general contexts. Our proofs
show the discrepancy between how easy it is to use algebra versus how
tedious it is to conduct analytical reasoning with inequalities and
calculus, and calls for a thorough examination on how to drastically
simplify analysis in these contexts.
\end{abstract}

\textsuperscript{$\dagger{}$}
These authors contributed equally to this work.

\textsuperscript{1} DIENS, École Normale Supérieure, CNRS, PSL
University, Paris, France

\textsuperscript{*} Correspondence:
\href{mailto:ryan.lahfa@ens.fr}{Ryan Lahfa
\textless{}ryan.lahfa@ens.fr\textgreater{}}

\hypertarget{introduction}{%
\section{Introduction}\label{introduction}}

\hypertarget{background-work}{%
\subsection{Background work}\label{background-work}}

The formalization of mathematics has seen a lot of projects:
\cite{wiedijkQEDManifestoRevisited}, \autocite{abelruffinicoq},
\autocite{feitthompsoncoq}, \autocite{buzzard2020perfectoids},
\autocite{lewis2019hensel}, \autocite{commelin2021witt}, most of them
treat of undergraduate mathematics and seldom of research-level
mathematics. In particular, the surrounding mathlib
\autocite{The_mathlib_Community_2020} formalization projects are
progressing at a fast pace, with Witt vectors
\autocite{commelin2021witt}, schemes \autocite{buzzard2021schemes}, the
Liquid Tensor Experiment \autocite{scholze2021liquid}. Yet, formalizing
research-level theories remain very difficult, especially when the
theory requires non-trivial metaprogrammation and tactics to simplify
proof terms.

In \autocite{buzzard2020perfectoids}, the definition of perfectoid
spaces is formalized entirely and required 33 files and more than 3000
lines of code which should have been in the mathematical library
(so-called \texttt{for\_mathlib} folder), upstreaming back such amount
of contributions is also a non-trivial problem
\autocite{van_Doorn_2020}. Their formalization also used ad-hoc
automation, notably with non-classical objects like algebraic structure
``with zero''.

In this paper, we will formalize the very start of an alternative
theory: Berkovich spaces. This paper follows those ideas and provides an
attempt to open up a formalization of Berkovich's young theory. To the
best of our knowledge, this formalization has never made its way in any
proof assistant.

We also show along the way that picking up a research-level theory
produces many undergraduate-level theorems the Lean mathematical library
lacks and how it can provide for better interfaces for further
formalizations.

\hypertarget{ostrowski-theorem-and-berkovich-spaces}{%
\subsection{Ostrowski theorem and Berkovich
spaces}\label{ostrowski-theorem-and-berkovich-spaces}}

This work will provide an in-depth view on the process of formalizing
Ostrowski theorem and its variants. In this section, we will first
re-introduce the mathematical contents. In section
\ref{sec:first_attempt}, we detail our bruteforce attempt to formalizing
the basic version of the theorem with minimal tooling. In section
\ref{sec:smart_attempt}, we use lessons learnt from the previous section
to generalize our tooling so that the Ostrowski theorem and its variants
can be derived while reusing as much as possible the steps and
arguments. In section \ref{sec:conclusion}, we provide our feedback on
the process and discuss future work to improve such formalizations and
this work.

The core objects of Ostrowski's theorem are \textbf{absolute values}:

\begin{definition}[absolute value] \label{def:absolute_value}
    An absolute value on a ring $R$ is a function $\abs{\cdot}: R \to \R$ such that
    \begin{enumerate}
        \item{} $\forall x \in R, \abs{x} = 0 \iff x = 0$
        \item{} $\forall x, y \in R, \abs{xy} = \abs{x} \abs{y}$
        \item{} $\forall x, y \in R, \abs{x + y} \le \abs{x} + \abs{y}$
    \end{enumerate}
\end{definition}

The usual absolute value is an absolute value with respect to Definition
\ref{def:absolute_value}.

These objects allow to build a completion of \(\Q\) in an algebraically
interesting way. The usual completion of \(\Q\) is \(\R\), and is
obtained with the usual absolute value. Absolute values retain just the
right amount of properties of the usual absolute value to show
\emph{both} analytical and algebraic interest.

In this paper, we focus on the following class of absolute values:

\begin{definition}[$p$-adic absolute value] \label{def:padic_abv}
    With $p \in \N$ prime, we denote $\abs{\cdot}_p$ the $p$-adic absolute value on $\Z$,
    where $\textrm{v}_p(k)$ is the multiplicity of $p$ in $k$:
    \begin{equation*}
        \abs{k}_p = p^{-\textrm{v}_p(k)}
    \end{equation*}
\end{definition}

The superclass of \(p\)-adic absolute values is the class of the
\emph{non-Archimedean absolute value}:

\begin{definition}[non-Archimedean absolute value] \label{def:nonArchimedean}
    An absolute value $\abs{\cdot}$ is called \emph{non-Archimedean} when the following holds:
    \begin{equation*}
        \forall x, y \in R, \abs{x + y} \le \max\left(\abs{x}, \abs{y}\right)
    \end{equation*}
\end{definition}

A natural question is to classify all absolutes values over \(\Q\),
which are classified \emph{up to equivalence}:

\begin{definition}[equivalence] \label{def:abv_equiv}
    Two absolute values $\abs{\cdot}_1$ and $\abs{\cdot}_2$ on a ring $R$ are said to be \emph{equivalent} when
    for some $\alpha > 0$ we have $\forall x \in R, {\abs{x}_1}^{\alpha} = \abs{x}_2$.

    When this holds, we write $\abs{\cdot}_1 \sim \abs{\cdot}_2$.
\end{definition}

It is noteworthy that equivalent absolute values are topologically
equivalent: this turns Ostrowski's theorem into a bridge between algebra
and analysis, completely classifying the absolute values on \(\Q\).

\begin{theorem}[Ostrowski] \label{target:ostrowski}
    Given $\lambda: \Q \to \Q$ an absolute value over $\Q$, either $\lambda \sim \abs{\cdot}$, either there is some $p \in \PR$ such that $\lambda \sim \abs{\cdot}_p$.
\end{theorem}

Such a theorem shows there is an alternative to the completion of \(\Q\)
by taking a prime number \(p\) and completing using \(p\)-adic absolute
value, giving arise to \(\Q_p\), and that these completions are the only
alternatives to the usual one.

Ostrowski's theorem plays an interesting role in Berkovich space theory
to completely determine the structure of the Berkovich spectrum of
integers: \(\mathcal{M}(\Z)\), which is the set of all norms over \(\Z\)
equiped with a certain topology.

Note that Ostrowski theorem has many variants where we can extend it to
fields like \(\F[X]\) or more complex structures. We will explore in
this work how a formalization of multiple variants can be obtained
efficiently.

For a more in-depth presentation of Berkovich space theory, refer to
\autocite{ducrosBerkovichSpacesApplications2015} or
\autocite{temkinIntroductionBerkovichAnalytic2015}.

\hypertarget{naive-formalization}{%
\section{\texorpdfstring{Naive formalization
\label{sec:first_attempt}}{Naive formalization }}\label{naive-formalization}}

To understand the challenges behind Ostrowski theorem being formalized,
we attempted a bruteforce formalization over \(\Q\) based on
\autocite{ruiterOstrowski}.

The resulting proof is easily understandable, only basic mathematical
tooling was needed as in the original proof: Bézout's identity, simple
limits and calculus.

Yet this proof does not fit the standard of formalized mathematics: it
is far too long and would greatly benefit from:

\begin{itemize}
\tightlist
\item
  extraction of lemmas, and generalization of most parts,
\item
  automation: most of the proof is calculus and could be automated with
  the right tactics and systems.
\end{itemize}

Concretely, the core lemma of this first attempt is around 200 lines
long. It is built mainly with the \texttt{obtain} keyword, which is the
formal equivalent of saying ``let us now show that \ldots''. This
construct allows us to stay close to the intuition but led us to longer
proofs, like in the toy example that follows.

For instance, one would start the proof of the bounded case by ``let us
first show that there is some \(n \in \N, n > 0\) such that
\(\abs{n} < 1\)'' (in this context, \(\abs{\cdot}\) is a nontrivial
bounded absolute value). To quickly prove this statement on a piece of
paper, we may say that:

\begin{itemize}
\tightlist
\item
  assuming \(\forall n \in \N^*, \abs{n} \ge 1\), then
  \(\forall n \in \N^*, \abs{n} = 1\) (\(\abs{\cdot}\) is bounded),
\item
  this is absurd because by hypothesis, \(\abs{\cdot}\) is nontrivial.
\end{itemize}

Following this exact scheme, our formalized proof starts with the
following :

\begin{lstlisting}
obtain ⟨ n, zero_lt_n, abvn_lt_one ⟩: ∃ n: ℕ, 0 < n ∧ abv n < 1,
{ /- 18 lines omitted -/ }
\end{lstlisting}

Suddenly, a two line ``human'' proof came out as a 18 lines long
formalized version. In fact, what we really did when we proved this
property in two sentences was:

\begin{itemize}
\tightlist
\item
  proceed by \emph{reducio ad absurdum},
\item
  realize that \(\abs{\cdot}\) is equal to \(1\) everywhere on \(\N\),
\item
  prove it by bounding the values of \(\abs{\cdot}\) using the suitable
  hypotheses,
\item
  realize that this is actually enough to prove that \(\abs{\cdot}\) is
  trivial,
\item
  show the contradiction by recalling our hypothesis : \(\abs{\cdot}\)
  is nontrivial.
\end{itemize}

Formalizing our two-liner required getting into punctilious details, and
even further formal considerations when detailing the very informal
``realize that \ldots''. Our readers can easily imagine how a handful of
calculations became a 200 lines formal proof for the core lemma.

\hypertarget{pursuing-a-general-enough-point-of-view}{%
\section{\texorpdfstring{Pursuing a general enough point of view
\label{sec:smart_attempt}}{Pursuing a general enough point of view }}\label{pursuing-a-general-enough-point-of-view}}

Naturally, the previous proof lacked of generality and contained too
much irrelevant detail which translated into bothersome ad-hoc
statements, so we adopted two objectives from this experience:

\begin{itemize}
\tightlist
\item
  as much as possible, make Ostrowski theorem a natural consequence from
  the general theory and allow for interesting generalizations,
  e.g.~Ostrowski over \(\F[X]\),
\item
  see how to fit parts of this general theory in the Lean mathematical
  library, so it can benefit other users.
\end{itemize}

Our intuition is a synthetic point of view is more suitable for
formalization than an analytic approach. Therefore, we went looking for
the adequate algebraic theories to support our goals.

We take inspiration from \autocite{artinAlgebraicNumbersAlgebraic2005}
presentation of Ostrowski theorem and transform the approach in a
suitable way for formalization.

\hypertarget{section:core_theory}{%
\subsection{Core of the theory}\label{section:core_theory}}

For this presentation, we will use \(R\) a principal ideal domain (PID).

The core idea is to keep an algebraic point of view and develop some
tools to characterize the behavior of bounded absolute values on general
rings (Definition \ref{def:our_boundness}).

\begin{definition} \label{def:our_boundness}
    Given $\abs{\cdot}: R \to \R$ an absolute value, $\abs{\cdot}$ is said bounded when:
    \begin{equation*}
        \forall x \in R, \abs{x} \leq 1
    \end{equation*}
\end{definition}

Note that this is equivalent to the usual definition of boundedness
(existence of some upper bound):

\begin{itemize}
\tightlist
\item
  if \(\abs{\cdot}\) is bounded, then \(1\) is an upper bound,
\item
  otherwise there is some \(x\) such that \(\abs{x} > 1\), then
  \(\abs{x^n} \xrightarrow[n \to +\infty]{} +\infty\) and
  \(\abs{\cdot}\) has no finite upper bound.
\end{itemize}

Furthermore, we define the \emph{trivial absolute value} as the function
that maps \(0\) to \(0\) and any other element to \(1\).

We will need one extra lemma for the core theorem, stating that an
absolute value is bounded over \(\N\) if and only if it is
non-Archimedean:

\begin{lstlisting}[label={contrib:nonArchimedean_iff_integers_bounded}]
theorem nonArchimedean_iff_integers_bounded
  {α} [comm_ring α] [nontrivial α] (abv: α → ℝ) [is_absolute_value abv]:
  (∃ C: ℝ, 0 < C ∧ ∀ n: ℕ, abv n ≤ C) ↔ (∀ a b: α, abv (a + b) ≤ max (abv a) (abv b))
\end{lstlisting}

Proving this lemma revealed to be challenging: on the paper, it takes at
most a dozen of lines, but the formalization took around 200 lines. The
reasons are the same as in section \ref{sec:first_attempt}. We have
isolated a corner of the theory where calculus cannot be avoided, like
we moved the problem that lied in section \ref{sec:first_attempt} from
one place to another. As future work, these lines would greatly benefit
from new calculus tactics.

The main theorem is \texttt{abv\_bounded\_padic}, which states that a
non-trivial bounded absolute value on a principal ideal domain \(R\) is
a \(p\)-adic absolute value for some prime \(p\) of \(R\).

\begin{lstlisting}[label={contrib:abv_bounded_padic}]
theorem abv_bounded_padic {α} [integral_domain α] [is_principal_ideal_ring α]
    [normalization_monoid α]
    (abv: α → ℝ) [is_absolute_value abv]
    (bounded: ∀ a: α, abv a ≤ 1)
    (nontrivial: ∃ a: α, a ≠ 0 ∧ abv a ≠ 1):
      ∃ (p: α) (p_prime: prime p), abvs_equiv abv (sample_padic_abv p p_prime)
\end{lstlisting}

The typeclasses \lstinline{[integral_domain α]} and
\lstinline{[is_principal_ideal_ring α]} ensure that \(\alpha\) is a
principal integral domain (PID).

\lstinline{[normalization_monoid α]} means that the elements of
\(\alpha\) admit a normal form (say, in \(\Z\), the positive integers,
and in \(\K[X]\), the monics). This is required by some of the lemmas we
use, but can be omitted for the scope of this paper.

\texttt{abvs\_equiv} is the relation of equivalence between absolute
values.

\texttt{sample\_padic\_abv\ p\ p\_prime} is an \(p\)-adic absolute value
(\texttt{p\_prime} is a proof that \(p\) is indeed prime).

Keeping in mind that according to
\lstinline{nonArchimedian_iff_integers_bounded}, \(\abs{\cdot}\) is
non-Archimedian, the strategy to prove the core lemma
(\texttt{abv\_bounded\_padic}) is as follows:

\begin{itemize}
\tightlist
\item
  Take \(\{ x \in R \mid \abs{x} < 1 \}\), this is a prime ideal of
  \(R\);
\item
  As \(R\) is a PID, there is some prime \(p \in R\) that generates the
  previous set;
\item
  Now, it is sufficient to prove the equivalence between \(\abs{\cdot}\)
  and \(\abs{\cdot}_p\) to finish;
\item
  By the primes extensionality lemma (see \ref{section:a_lemma}), it
  suffices to prove there is some \(\alpha > 0\) such that for all prime
  \(q \in R\), \(\abs{q}^{\alpha} = \abs{q}_p\);
\item
  To clear this goal, a case analysis on whether \(p\) and \(q\) are
  associated is enough and helps to find the suitable \(\alpha\) in
  terms of logarithms of absolute values of \(p\).
\end{itemize}

The core lemma is easy to prove as it is the result of composable and
reusable lemmas and proves our point regarding the need of finding
general enough abstractions so that the proofs tend towards an
assembling game.

\nopagebreak[4]

\hypertarget{section:a_lemma}{%
\subsection{A lemma}\label{section:a_lemma}}

We also encountered a very useful extensionality lemma for morphisms
over monoids with zero, of which we give the Lean definition:

\begin{lstlisting}[label={contrib:ext_hom_primes}]
theorem ext_hom_primes {α} [comm_monoid_with_zero α] [wf_dvd_monoid α]
  {β} [monoid_with_zero β]
  (φ₁ φ₂: monoid_with_zero_hom α β)
  (h_units: ∀ u: units α, φ₁ u = φ₂ u)
  (h_irreducibles: ∀ a: α, irreducible a → φ₁ a = φ₂ a):
    φ₁ = φ₂
\end{lstlisting}

\lstinline{[monoid_with_zero β]} states that \(\beta\) is a monoid that
contains a ``zero'', \emph{i.e.} an absorbing element. These objects may
seem peculiar to a mathematician, but are useful in the context of
formalized mathematics. We will not discuss the use of ``monoids with
zero'', as they are outside of the scope of this article.

\lstinline{[comm_monoid_with_zero α]} further states that \(\alpha\) is
commutative.

\lstinline{φ : monoid_with_zero_hom α β} states that \(\varphi\) is a
homomorphism of monoid with zero with source \(\alpha\) and target
\(\beta\).

\lstinline{[wf_dvd_monoid α]} states that the division on \(\alpha\) is
a well-founded order. This is key to the lemma: we only need to proceed
by induction. This makes this lemma apply well to principal ideal
domains, because the division in such rings the inclusion is
well-founded.

Mathematically, this lemma states that if

\begin{itemize}
\tightlist
\item
  \(R\) is a principal ideal domain
\item
  Two multiplicative functions agree on the units of \(R\) and on its
  primes
\end{itemize}

Then, they coincide everywhere.

This nontrivial lemma may be useful to anyone working with
multiplicative functions and was added to mathlib
\autocite{The_mathlib_Community_2020}. We therefore fulfilled one of our
two goals: formalizing mathematics which may be useful to future users.

We brought the problem into statements about multiplicative functions,
but have yet to lift our original statement which has a valuation flavor
in these terms. Note that valuations and multiplicative functions
(actually, homomorphisms) are unfortunately very different objects in
Lean: the former are just functions that are refined using a typeclass,
while the latter are \emph{structures} (in a nutshell, tuples containing
objects and proofs). This implies that switching from the valuation
point of view to homomorphisms and back is cumbersome. To solve this
problem, we had to write some boilerplate to bridge this gap. As future
work, it might be possible to automate the process of switching of point
of view on this kind of objects, certainly through meta-programming
\autocite{commelin2021witt}.

\hypertarget{application-ostrowski-on-mathbbq}{%
\subsection{\texorpdfstring{Application: Ostrowski on
\(\mathbb{Q}\)}{Application: Ostrowski on \textbackslash mathbb\{Q\}}}\label{application-ostrowski-on-mathbbq}}

Once the core lemmas are laid out, Ostrowski's theorem on \(\Z\) is
almost immediate. Now, obtaining it over \(\Q\) requires the extension
of absolute values to the entire field. In theory, it is also almost
immediate because of the multiplicative property of absolute values.

In practice, some manual work remained to lift results from \(\Z\) to
\(\Q\), yet, this is not a failure of the previous goal to pursue a
general enough theory but rather what we would believe to be a lack of
automation in the proof assistant which could be alleviated by
meta-programming. That being said, we did not pursue this venture and
test our hypothesis and will discuss it later.

\hypertarget{application-ostrowski-on-mathbbfx}{%
\subsection{\texorpdfstring{Application: Ostrowski on
\(\mathbb{F}[X]\)}{Application: Ostrowski on \textbackslash mathbb\{F\}{[}X{]}}}\label{application-ostrowski-on-mathbbfx}}

We proved a statement that is slightly less powerful in spirit, in that
it does not actually cover \emph{all} the possible absolute values, but
only the absolute values that are trivial on \(\mathbb{F}\).

\begin{theorem}[Ostrowski variant] \label{contrib:ostrowski_variant}
    Given $\abs{\cdot}$ an absolute value on $\F[X]$, trivial on $\F$.
    Exactly one of the following is true:

    \begin{itemize}
    \item $\abs{\cdot}$ is bounded and for some prime $p \in \F[X]$, $\abs{\cdot} = \abs{\cdot}_p$.
    \item $\abs{\cdot}$ is equivalent to the degree.
    \end{itemize}
\end{theorem}

Comforting our intuition, both cases were straightforward reusing the
tools in section \ref{section:core_theory}.

\hypertarget{conclusion}{%
\section{\texorpdfstring{Conclusion
\label{sec:conclusion}}{Conclusion }}\label{conclusion}}

\vspace*{-0.2em}

\hypertarget{results}{%
\subsection{Results}\label{results}}

We wanted to examine the difficulties of formalizing Ostrowski's theorem
which constitutes the first step towards Berkovich space theory
formalization.

With a bruteforce method, we encountered many tedious computations of
analytical nature which led us to hide all the complexity inside algebra
which was easier to handle in the proof assistant. The second part
presents an approach which worked effectively and gave us with fewer
efforts more theorems and provided us with insights on how to pursue the
generalization.

Nevertheless, this suggests that calculus and analysis might benefit
from a local framework which might help with their manipulation,
non-standard analysis seems a promising avenue already explored in
Isabelle/HOL with \autocite{fleuriot2000mechanization}, but also, the
Lean theorem prover in its version 4 might help with its treatment of
coercions and performance improvements \autocite{Lean4_2021}.

\vspace*{-0.2em}

\hypertarget{outlook}{%
\subsection{Outlook}\label{outlook}}

We notice formalization is not only a process that helps you verify a
proof but also to understand and provide insights on results surrounding
a theory and sustain the improvements on the system being used beyond
classical computer science aspects like performance or user experience.

In particular, we identified pain points using the Lean theorem prover
which constitutes interesting future works, namely automation to:

\begin{itemize}
\tightlist
\item
  combine analysis and inequalities/equalities reasoning, e.g.~taking
  limits on a side or both sides,
\item
  bridge points of view or even theories, \emph{e.g.} the former
  discussion on valuations and homomorphisms.
\end{itemize}

Despite these bothersome points, we found that adopting a synthetic
approach alleviates us from most of the hardships that we encountered
with an analytic approach.

Finally, now that Ostrowski theorems are formalized, it is possible to
produce the basic objects of Berkovich space theory, notably the
Berkovich spectrum and give a non-trivial example: \(\mathcal{M}(\Z)\).

\vspace*{-0.5em}

\printbibliography[title=References]

\end{document}
